\section{Introduction}\label{sec:intro}
% Your introduction should contain an overview of the work you were assigned, as well as a few sentences putting the work into a larger perspective. You should also give a quick description of how the report is organized (as is done below).
This report is organized into six major sections: introduction, part one, part two, part three, conclusion, and appendix. The introduction outlines the work. Part one describes the problem and its mathematical modelling. Part two derives various results. Part three presents the results as answers to the given tasks. The conclusion concludes our work. In the appendix we've supplied various sources such as code, and the original asking paper.

% You should of course put most of the work into doing good work in the lab and then presenting it in the report. When presenting your work in the report, both content and presentation/layout matters. Since your only way of communicating your good effort in the lab is through writing about it here, the way you write about it is essential. This means that even if you have the very best controller but describe it poorly, you will probably not be rewarded for the good results. A plot showing perfect control is worth very little if it is not accompanied by a clear description of what it represents.

% Layout is naturally less important than content, but it still matters. You can think of report writing like selling an apartment; when you present your apartment for potential buyers you will of course clean the apartment and make it good looking. How clean the apartment is does of course not determine its value, but it is still important since it influences the subjective value your buyers will put on the apartment.

\subsection{Other Comments}
% This report is organized as follows: \Cref{sec:prob_descr} contains some equations relevant for TTK4135, and some tips on how to create illustrations. Several \LaTeX{} tips can be foundin \Cref{sec:latex_tips}, such as how to create a table and matrix equations. \Cref{sec:figures} contains some advice on using plots from MATLAB\@. The closing remarks are in~\Cref{sec:conclusion}, respectively. \Cref{sec:matlab} contains a MATLAB file while \Cref{sec:simulink} shows an example Simulink diagram. The Bibliography can be found at the end, on page~\pageref{sec:bibliography}.
